\documentclass[a4paper]{article} % Plantilla 

%Uso de paquetes 
\usepackage[utf8]{inputenc} % Codifica el archivo en utf-8
\usepackage[spanish]{babel} % Codifica el archivo en utf-8
\usepackage{graphicx} % inserción de imagenes.  
\usepackage[table,xcdraw]{xcolor} %Definicion De Colores.
\usepackage[most]{tcolorbox} % Para la creacion de cuadros.
\usepackage[margin=2cm]{geometry} % Margen del informe ne general
\usepackage{fancyhdr} % Define el estilo de la pagina.
\usepackage[hidelinks]{hyperref} % Gestion de hipervinculos.
\usepackage{parskip} %Arreglo de la tabulacion del documento.
\usepackage[figurename=c3lis]{caption} %Cambiar el nombre del caption de las fotos.
\usepackage{smartdiagram} % Para el insertado de diagramas.
\usepackage{listing} % Para la inserción de codigo en el documento.
\usepackage{zed-csp} % Para la inserción de esquemas.

% Declaracion de colores
\definecolor{green}{HTML}{CF010B}
\definecolor{black}{HTML}{000000}

% Declaracion de variables.
%Imagenes.
\newcommand{\logMachine}{img/logCheese.jpg}
\newcommand{\logo}{img/tryhackme_log.png}
%Variables de entorno.
\newcommand{\machineName}{Cheese} % Nombre de la maquina.
\newcommand{\startDate}{7 De Octubre Del 2024}

%Adicionales
% \addto\captionsspanish{\renewcommand{\contentsname}{Index}} Cambia la variable por defetcto que marca el indice.
\setlength{\headheight}{80pt}
\pagestyle{fancy}
\fancyhf{}
\lhead{\includegraphics[width=4cm]{\logo}}\rhead{\includegraphics[height=3cm,keepaspectratio]{\logMachine}}
\renewcommand{\headrulewidth}{3pt}
\renewcommand{\headrule}{\hbox to\headwidth{\color{green}\leaders\hrule height \headrulewidth\hfill}} % Cambie el color de la lineas del indice.
\renewcommand{\lstlistingname}{codigo} % Para cambiar los captions de los. codigos.


\definecolor{codegreen}{rgb}{0,0.6,0}
\definecolor{codegray}{rgb}{0.5,0.5,0.5}
\definecolor{codepurple}{rgb}{0.58,0,0.82}
\definecolor{backcolour}{rgb}{0.95,0.95,0.92}

\lstdefinestyle{mystyle}{ % Estilo por la importacion de un cuadro
    backgroundcolor=\color{backcolour},   
    commentstyle=\color{codegreen},
    keywordstyle=\color{magenta},
    numberstyle=\tiny\color{codegray},
    stringstyle=\color{codepurple},
    basicstyle=\ttfamily\footnotesize,
    breakatwhitespace=false,         
    breaklines=true,                 
    captionpos=b,                    
    keepspaces=true,                 
    numbers=left,                    
    numbersep=5pt,                  
    showspaces=false,                
    showstringspaces=false,
    showtabs=false,                  
    tabsize=2
}



%Comienzo del documento.
\begin{document}
	\cfoot{\thepage}
	%----------------------------------------
	%Creacion de la portada.
	\begin{titlepage}	
	\center
	\includegraphics[width=0.5\textwidth]{\logo}\par\vspace{1cm}
	{\scshape\LARGE\textbf{Informe Tecnico.}\par\vspace{0.3cm}}	
	{\Huge\bfseries\textcolor{green}{Maquina \machineName}}\par
	\vfill\vfill
	\includegraphics[width=\textwidth,height=10cm,keepaspectratio]{\logMachine}\par\vspace{1cm}
	\vfill
	\begin{tcolorbox}[colback=red!5!white,colframe=red!75!black]
		\center
		\LARGE{Este documento es confidencial y contiene informacion sensible.\\Esta informacion no deberia compartirse con terceros.}
	\end{tcolorbox}
	\vfill
	{\large\startDate\par}
	\vfill 
	\end{titlepage}
	%---------------------------------------
	% Comienzo del TOC
	\clearpage
	\tableofcontents
	\clearpage
	%---------------------------------------
	
	\section{Antecedentes}
	El presente documentos recoge los resultados de la auditoria realizad a la maquina {\textbf \machineName} de la plataforma \href{https://tryhackme.com}{\textbf{\color{blue}TryHackme}}.
	\vspace{0.4cm}
	\begin{figure}[h]
	\center
	\includegraphics[width=\textwidth]{img/information_machine.png}
	\caption{Detalles de la maquina.} %Texto debajo de la imagen. 
	\end{figure}
	
	\vspace{0.3cm}
	\begin{tcolorbox}[enhanced,attach boxed title to top center={yshift=-3mm,yshifttext=-1mm},
  		colback=blue!5!white,colframe=black!75!black,colbacktitle=black!80!black,
		title=Url=\bfseries,
 		boxed title style={size=small,colframe=red!50!black} ]
 		\center
		\href{https://tryhackme.com/r/room/cheesectfv10}{\color{black}{https://tryhackme.com/r/room/cheesectfv10}}
	\end{tcolorbox}

	\section{Objetivos}
	Conocer el estado de actual de el servidor de \textbf{\machineName}, enumerando posbiles vectores de explotacion y determinando el alcance de impacto, que un atancante podria ocasionar en un sistema en produccion.
	\subsection{Consideraciones} %Subseccion de la seccion.
	Una vez finalizada las jornadas de auditoria se llevara a cabo un serie de saneamientos y buenas practicas con el objetivo de securizar el servidor he evitar ser victima de un futuro ataque.

	\vspace{0.5cm}
	\begin{figure}[h]
	\begin{center}
	\smartdiagram[priority descriptive diagram]{
	Reconocimiento sobre el sistema,
	Explotacion de vulnerabilidades,
	Exploracion del sistema,
	Securizacion del sistema.
	}
	\end{center}
	\end{figure}
	\clearpage
	\section{Reconocimiento del sistema.}
	\subsection{Reconocimiento Inicial.}
	\vspace{0.3cm}
	Se comenzo realizando un analizis inicial sobre el sistema verificando que el sistema obejetivo que puertos tiene abiertos.

	\vspace{0.5cm}
	\begin{figure}[h]	
	\begin{center}
		\includegraphics[width=0.8\textwidth]{img/reconocimiento.png}
		\caption{Reconocimiento de puertos abiertos por el servidor.}
	\end{center}
	\end{figure}
	\vspace{0.2cm}
	
	Asi mismo para evitar ciertos falsos positivos se construyo en bash un script en bash para verificar con exactitud la cantidad de puertos abiertos en el servidor.
	\vspace{0.5cm}
	\begin{lstlisting}[language=Bash, caption=Script]
	#!/bin/bash
	
	for por in $(seq 1 6555);do
		time out 1 bash -c "echo > /dev/tcp/10.10.118.129/$port" >/dev/null && echo "$port/tcp" &
	done; wait 
	\end{lstlisting}
	\vspace{1cm}
	A travez del anterior script es posbile detectar los siguientes puertos.
	\begin{schema}{TCP}
	Puertos
	\where
	22, 80, 4444
	\end{schema}

	\vspace{1cm}
	Una vez finalizado el escaneo se detectaron la version y servicio que corrian bajo estos, representando a continuacion los puertos criticos los cuales fueron la continuidad a la explotacion del sistema.
	
	
	\clearpage
	\section{Explotacion de vulnerabilidades.}
	\subsection{Codigo de panel vulnerable a ataque de Inyeccion SQL}
	Posteriormente se empezo con el analizis de rutas existente por parte del puerto \textbf{80=HTTP} descubriendo que era susceptible a una ataque de inyeccion \textbf{SQL}.
	\vspace{0.5cm}
	\begin{figure}[h]
	\center	
		\includegraphics[width=0.8\textwidth]{img/vulnerabilidad.png}
	\end{figure}
		
\end{document}


